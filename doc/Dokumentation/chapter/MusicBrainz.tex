\section{MusicBrainz}
\label{sec:MusicBrainz}

MusicBrainz\footnote{http://musicbrainz.org/} ist ein Webservice ueber den Daten zu Alben und Kuenstlern abgerufen werden k�nnen. Die API erlaubt Abfragen. Dabei kann man entweder allgemein suchen dann bekommt man eine Ergebnisse liste die nach einen Score sortiert ist. Man kann aber auch speziell nach einer eindeutigen ID der MBID suchen. Unsere Vorstellung war es MusicBrainz zur Erstellung einer CD zu verwenden. Der Benutzer muss dazu mindestens das Album und den Kuenstler angeben. Diese Daten sind fest und werden von MusicBrainz nicht ver�ndern. Anhand dieser Daten wird versucht den K�nstler und das Album zu finden. War die Suche erfolgreich wird nach dem Erscheinungsjahr und Tracks geschaut. Diese werden in die Form eingetragen und k�nnen vom Benutzer noch bearbeitet, erg�nzt oder gel�scht werden bevor das Formular abgeschickt wird und die CD erstellt wird. Gleichzeitig wird eine Amazon Standard Identifikation Number (asin) gespeichert welche die CD eindeutig bei Amazon identifiziert. Dar�ber kann eine URL erstellt werden, �ber welche ein Cover heruntergeladen werden kann.Diese hat die Form http://images.amazon.com/images/P/<ASIN>.jpg, wobei ASIN irgendeine eindeutige Nummer darstellt. Ist ein Cover vorhanden wird das �ber die URL heruntergeladen, so das der Benutzer selbst kein Cover hochladen muss.
