\section{Tests}
\label{sec:Tests}
Wir haben das Standard Testframework von Rails verwendet um umfangreiche Unit Tests zu schreiben. Dies basiert auf der Idee des Test-Driven-Development. Also zuerst Tests schrieben, schauen wie sie fehlschlagen und dann die Funktionalit�t implementieren und pr�fen ob der Test erfolgreich ist. ie Unit Tests sichern unsere Grundfunktionalit�t. Des weiteren haben wir mit Cucumber Tests geschrieben. Mit Cucumber (Behavior-Driven-Development) ist es m�glich Tests in Prosaform zu schrieben. Das macht die Tests besser lesbarer auch f�r nicht Informatiker. Die Tests werden h�ufig als User Stories geschrieben in der Form: ''Als <Rolle>, m�chte ich <Ziel/Wunsch>, um <Nutzen>''. Umfangreiche Tests wie sie in einem realen Projekt gemacht werden sollten konnten wir aufgrund der beschr�nkten Ressourcen jedoch nicht durchf�hren. Rails eignet jedoch wunderbar f�r ein Test-Driven-Development da es ein Standard Testframwork integriert. 