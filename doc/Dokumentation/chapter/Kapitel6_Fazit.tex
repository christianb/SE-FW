\chapter{Zusamenfassung}
\label{sec:Zusammenfassung}
\section{Antonia Ziegler}
Insgesamt würde ich Ruby on Rails als ein sehr modernes und zukunftsorientiertes Framework mit einer grossen Community bezeichnen.
Die Entwicklungsumgebung lässt sich insgesamt gut auf verschiedenen Betriebssystemen aufsetzen. 
Wenn man sich an das gegebene MVC Prinzip hält, ist ein Ruby on Rails Projekt gut erweiterbar und wartbar. 
Was mir am besten gefällt ist die Möglichkeit, oft ziemlich einfach, Plugins einzubinden. Auf Grund der grossen Community gibt es viele Open Source Plugins, welche Funktionen bieten, wie z.B. die Pagenavigation. Diese wiederum sind gut erweiterbar oder anpassbar.
\section{Christian Sandvoss}
Ich finde, das man Railsls sehr gut verwenden kann um auch umfangreiche Webanwendung zu realisieren. Rails stellt ein Grundsystem bereit, das durch viele Pugins erweiterbar ist. Allerdings muss man dabei darauf achten ob das Plugin zur genutzten Rails Version kompatibel ist oder überhaupt noch weiter entwickelt wird. Des weiteren ist bei einigen Plugins die Dokumentation sehr kurz gehalten. Die Dokumentation von Rails ist dagegen, wie ich finde, sehr gut. Es gibt Tutorials zum Einstieg sowie eine aktive Community. Neu für mich war die Arbeit mit JavaScript bzw. JQuery. Es war für mich sehr interessant das Zusammenspiel von verschiedenen Sprachen (Ruby, HTML/CSS, JavaScript) kennenzulernen.

\section{Alexander Miller}
Durch die Realisierung des Projektes konnte ich persönlich feststellen welche Vor- und Nachteile Rails ausweist. Die Verwendung von Rails, können sogar komplexe RIA Web-Anwendungen realisiert werden. Die wesentlichen Prinzipien dieses Frameworks sind Konvention over Configuration und MVC Prinzip. Im Allgemeinen ist Rails sehr gut dokumentiert und hat eine grosse und vor allem aktive Community. Der Umfang an Möglichkeiten ist enorm gross, so dass viele Funktionalitäten problemlos implementiert werden können. Darüber hinaus besteht die Möglichkeit viele Module als Plug-Ins zu integrieren. Der Prozess der Integration von PlugIns ist trivial wenn dieser ausreichend Dokumentiert ist. Leider sind manche PlugIns schlecht dokumentiert, so dass die Auseinandersetzung manchmal viel Zeitaufwand erfordert. Insgesamt bin ich mit dem Framework sehr zufrieden, so dass es nichts dagegen spricht für die Realisierung einer weiteren Web-Applikation Rails zu verwenden.
\section{Christian Bunk}
Ich muss sagen das sich meine Einstellung zu Webframeworks sehr geändert hat. Ich hatte früher nie Lust auf Webentwicklung, da mir Frameworks wie Spring und Struts viel zu kompliziert, zu sperrig, ja sich einfach nicht schön an die Entwicklung von Web Applicationen angefühlt haben. Aber mit Ruby on Rails hat sich das sehr geändert. Schon die Programmiersprache Ruby macht das schreiben von Programmen sehr angenehm. Es ist eine elegante und moderne Sprache. Darauf ein Webframework aufzubauen erscheint nur logisch. Mit Ruby on Rails hat man das Gefühl das alles gut durchdacht ist. Durch das Prinzip Convention over Configuration ist es möglich mit wenig aufwand viel zu erreichen. Alles ist da wo es sein soll und wo es hingehört. Hält man sich an bestimmte Regeln des Frameworks macht die Entwicklugn einfach Spass. Das bedeutet aber nicht das die Entwicklung leicht ist. Im Gegenteil Rails ist ein Framework mit sehr grossem Funktionsumfang. Selbst in den 3 Monaten unserer Entwicklung haben wir in vielen Bereich nur an der Oberfläche gekratzt. Rails bietet eine wunderbare und umfangreiche Dokumentation. Aber um hier hinter alle Konzepte zu steigen braucht es einfach Zeit. Rails bietet mit Erweiterungen durch Module oder Plugins die Möglichkeit Funktionen in seine Webanwendung zu integrieren. Das ist gut solange man nicht mehr Funktionalität braucht als die jeweiligen Plugins bieten. Schwierig wird es wenn man doch eigene oder zusätzliche Funktionen benötigt. Dann muss man versuchen um das Plugin herum zu entwickeln, da meist genaue Dokumentation für die sehr speziellen Fälle fehlen. Mit Rails habe ich auch das Routing, also das mappen von URLs auf bestimmte Methoden in einem Controller, besser verstanden. Das Konzept von Active Record hat mich sehr überzeugt, da hier SChnittstellen und Datenbanken optimal verschmelzen und nicht mehr an jeder Stelle im Code die Datenbank kryptisch ausgelesen oder konfiguriert werden muss. Es wird einfach ein Model definiert, über welches man die Attribute definiert. Diese werden dann automatisch in die Datenbank eingetragen. Rails eignet sich sehr gut für verteilte Entwicklung. Es werden weder Lizenzen benötigt, noch wird dem Programmierer eine IDE aufgezwungen. Alles kann in einem einfachen Texteditor entwickelt werden. Mehr als einen Browser und ein Terminal braucht man dann nicht. Entgegen manchen Vorstellungen das Rails einfach sei muss ich wiedersprechen. MAn braucht viel Zeit um die Konzepte sich zu eigen zu machen. Auch braucht man (wie bei jedem anderen Framework) Zeit um Funktionalitäten zu programmieren. Ich denke eh nicht das man Frameworks nach leichter und schwerer Kategorisieren kann. Ich denke das ich mit Rails wertvolle Erfahrungen gesammelt habe die mir im meinem späteren Berufsleben nützlich sein werden. Ich denke auch das ich mit Rails noch öfter in Zukunft in berührung kommen werde. Rails muss man einfach erlebt haben.