\section{File Upload (Paperclip)}
\label{sec:File Upload (Paperclip)}
In der Webanwendung sollen zwei Arten von Dateien hochgeladen werden k�nnen: Bilder und Audio. Um Dateien zum Server zu laden stehen f�r Rails zwei Plugins in Rails zur Verf�gung Attachment-fu und Paperclip\footnote{https://github.com/thoughtbot/paperclip}. Wir haben uns f�r Paperclip entschieden das etwas flexibler als Attachment-fu ist. M�chte man Bilder mit Paperclip zum Server hochladen wird zus�tzlich noch das Tool ImageMagick ben�tigt, welches in der Lage ist verschiedene Versionen eines Bildes zu erstellen sowie die Meta-Informationen auszulesen. Im Model wird definiert welche Art von Datei hochgeladen werden kann. So sollen in f�r den User als auch f�r die CD's Bilder hochgeladen werden k�nnen. Dazu m�ssen in der Datenbank zus�tzliche Spalten angelegt werden. Paperclip speichert die hochgeladenen Daten nicht in der Datenbank sondern auf dem Dateisystem. In der Datenbank wird lediglich die URL zu der Datei gespeichert. Das macht den ganzen Prozess wesentlich performanter, da die Datenbank nicht so vollgestopft wird. Das hochladen von Audio funktioniert fast genauso, nur das man hier auf andere Mime-Types pr�ft. Es hat sich gezeigt dsa verschiedene Browser und auch verschiedene Plattformen unterschiedlich mit den Mime-Types umgehen. Mit Paperclip kann sehr schnell ein Datei Upload eingerichtet werden. 
Im folgenden ist zu sehen wie ein Attachment dem Model hinzugef�gt werden kann inklusive Validierung. Im folgenden ist f�r das Model User die Anpassung zu sehen um ein Cover hochzuladen. Die entsprechenden Attribute m�ssen in der Datenbank in der entsprechenden Tabelle �ber eine Migration eingef�gt werden.
\lstinputlisting[language=Ruby]{chapter/file_upload.rb}