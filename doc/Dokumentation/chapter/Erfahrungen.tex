\chapter{Erfahrungen vor dem Projekt}
In diesem Kapitel beschreiben die Projektteilnehmer ihre Erfahrungen mit Webframeworks vor dem Projekt.
\section{Antonia Ziegler}

\section{Christian Sandvoss}

\section{Alexander Miller}
Vor der Lehrveranstaltung SE-FW hatte ich keine Erfahrungen mit Ruby on Rails gehabt. Es war mir klar auf welchen Ansätzen dieses Framework basiert. Um mehr darüber zu erfahren und die einzelnen Vorteile durch ein Projekt zu erkennen, habe ich in meiner Gruppe ebenfalls für Ruby on Rails abgestimmt. Meine Erwartungen an dieses Framework sind: Flexibilität, Einfachheit, Erweiterbarkeit und schnelle Implementierung.

\section{Christian Bunk}
Ich habe bisher nur oberflächlich mit web frameworks gearbeitet. 
Im fünften Semester Bachelor haben wir an der Plattform NetBase Education von Prof. Langebein gearbeitet. 
Dort wurde Struts verwendet. Aufgrund der grösse des Projekts war die Arbeit mit Struts sehr umständlich. 
Viele Arbeiten mussten von Hand erledigt werden. Die Arbeit mit Web Framworks war für mich eine eher lästige angelegenheit. 
In meinem Praktikum bei ART+COM habe ich dann einiges über Ruby on Rails gehört. 
Dort wurde mir dann klar das ich mich unbedingt dieses Framework in mein Reportoir aufnehmen muss. 
Leider hatte ich seitdem noch keine Gelegenheit etwas praktisches damit umzusetzen. 
Aus diesem Grund wollte ich in dem Fach Systementwicklung und Frameworks etwas mit diesem Framework machen.
