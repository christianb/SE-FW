\section{Tests}
Wir haben das Standard Testframework von Rails verwendet um umfangreiche Unit Tests zu schreiben. Dies basiert auf der Idee des Test-Driven-DEvelopment. Also zuerst Tests schrieben, schauen wie sie fehlschlagen und dann die Funktionalität implementieren und prüfen ob der Test erfolgreich ist. ie Unit Tests sichern unsere Grundfunktionalität. Des weiteren haben wir mit Cucumber Tests geschrieben. Mit Cucumber (Behavior-Driven-Development) ist es möglich Tests in Prosaform zu schrieben. Das macht die Tests besser lesbarer auch für nicht Informatiker. Die Tests werden häufig als User Stories geschrieben in der Form: "Als <Rolle>, möchte ich <Ziel/Wunsch>, um <Nutzen>". Umfangreiche Tests wie sie in einem realen Projekt gemacht werden sollten konnten wir aufgrund der beschränkten Ressourcen jedoch nicht durchführen. Rails eignet jedoch wunderbar für ein Test-Driven-Development da es ein Standard Testframwork integriert. 