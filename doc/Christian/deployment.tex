Da die Anwendung auf einem Webserver zur Verfügung gestellt werden soll, wurde dieser eingerichtet\footnote{http://kallisto.f4.htw-berlin.de}. Dafür wurde auf den uns zur Verfügung stehenden Server, als erstes die benötigte Software installiert. Um überhaupt Rails Anwendungen ausführen zu können, wurde Ruby in der Version 1.9.2 und Rails 3.1.1 installiert. Anschließend wurde der Apache Webserver installiert. Dieser wurde entsprechend konfiguriert sowie eine PostGreSQL Datenbank installiert. Desweiteren wurde, Capistrano ein OpenSource Deployment-Tool für Rails-Anwendungen, installiert. Dabei handelt es sich um eine Software für das automatisierte Ausführen von Aufgaben auf einem oder mehreren entfernten Servern. Das zentrale Prinzip des Tools ist die vollständige Automatisierung des Verteilungsprozesses. Somit sind die einzelnen Schritte in einem zusammengefasst, wie: Auschecken der Software aus der Versionskontrolle, Ausführen der Unit-Tests, Übertragung der Software auf die Ziel-Server, Aktualisierung der Datenbanken und Neustart des Webservers. Diese Schritte wären einzeln ausgeführt sehr zeitintensiv, fehleranfällig und zu dem auf Dauer langweilig, vor allem da in einem agilen Entwicklungsprozesses dies relativ häufig ausgeführt werden muss.