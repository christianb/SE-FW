\chapter{Einleitung}
\label{sec:Einleitung}

In der Softwaretechnik wird unter dem Begriff Framework ein Ordnungsrahmen verstanden, innerhalb dessen die Entwickler eine Anwendung erstellen. Das Ziel ist die Definition einer einheitlichen Struktur mittels komponentenbasierten Entwicklungsans�tzen sowie die Vereinfachung der Entwicklung durch die Verwendung von zur Verf�gung stehenden Frameworks-Bestandteilen. Im Laufe der Lehrveranstaltung ''Systementwicklung und Frameworks'' wurden die Prinzipien der unterschiedlichen Frameworks wie zum Beispiel EJB und .NET vorgestellt und diskutiert. Als Pr�fungsleistung muss ein Projekt erfolgreich realisiert werden, wobei jede Gruppe von vier bis f�nf Personen die Anwendung mit einem ausgew�hlten Framework entwickeln m�ssen. 

In der vorliegenden Ausarbeitung handelt es sich um eine Dokumentation zum Projekt, wobei eine CD-Tauschb�rse konzipiert und implementiert wurde. Wie in der Aufgabenstellung bzw. im Pflichtenheft definiert (Anhang A.1) wurde eine Rich-Applikation (RIA) modelliert und entwickelt, wobei die meisten Funktionalit�ten nicht von Hand, sondern durch die Verwendung von Framework-Komponenten realisiert wurden. Die Gruppe besteht aus vier Personen (Christian Bunk, Alexander Miller, Christian Sandvo�, Antonia Ziegler) und entwickelt das Projekt mit dem Framework ''Ruby on Rails''. 

Ruby on Rails ist ein Framework f�r Webapplikationen, welches in der Programmiersprache Ruby entwickelt wurde. Model-View-Controller (MVC) Architektur erm�glicht eine Isolation der Businesslogik von der graphischen Benutzeroberfl�che und gew�hrleistet das ''don't repeat yourself'' (DRY) Prinzip. 

Der SourceCode wird mit Git (https://github.com/christianb/SE-FW) verwalten. Hierzu wird die online Plattform GitHub als zentrales Repository verwendet, die sowohl die aktuellen Aufgaben (Issues) als auch die Meilensteine verwaltet. Um eine Arbeitsgrundlage zu erschaffen, haben alle Gruppenmitglieder sich das grundlegende Wissen �ber das ausgew�hlte Framework angeeignet. Wie die geforderten Funktionalit�ten im Einzelnen implementiert sind, kann den folgenden Kapiteln entnommen werden. An dieser Stelle muss noch erg�nzt werden, dass alle gestellten Anforderungen erf�llt wurden.
